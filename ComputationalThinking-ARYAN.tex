   
\documentclass[11pt]{article}
\renewcommand{\baselinestretch}{1.05}
\usepackage{amsmath,amsthm,verbatim,amssymb,amsfonts,amscd, graphicx}
\usepackage[mathscr]{euscript}
\usepackage{graphics}
\usepackage{amsmath}
\usepackage{cite}
\usepackage{graphicx}
\graphicspath{ {/home/adamryan/Documents/MySQL/} }
\usepackage{hyperref}
\hypersetup{
	colorlinks=true,
	linkcolor=blue,
	filecolor=magenta,      
	urlcolor=cyan,
}
\topmargin0.0cm
\headheight0.0cm
\headsep0.0cm
\oddsidemargin0.0cm
\textheight23.0cm
\textwidth16.5cm
\footskip1.0cm
\theoremstyle{plain}
\newtheorem{theorem}{Theorem}
\newtheorem{corollary}{Corollary}
\newtheorem{lemma}{Lemma}
\newtheorem{proposition}{Proposition}
\newtheorem*{surfacecor}{Corollary 1}
\newtheorem{conjecture}{Conjecture} 
\newtheorem{question}{Question} 
\theoremstyle{definition}
\newtheorem{definition}{Definition}

 \begin{document}

\title{COMP47340: The Digital Transformation of Retail and the Ethical Implications}
\author{Adam Ryan}
\maketitle

\section{Abstract}\label{section:Abstract}

Drawing on experience working as a data scientist within an Irish fashion retailer, and the material discussed within COMP47340, the author examines the topics of how technology has impacted retail with a particular emphasis on customer analytics $($CT5-ComputersProblemSolving, slide 21$)$, views in detail the application of text analytics within retail $($CT7-What Lies Beneath, slide 39$)$, and discusses some ethical concerns of using customer analytics within a retail context $($CT9-Machine Ethics, slide 11$)$.\\
\\
\textbf{Section Word Count}: $86$ words.\\
\textbf{Cumulative Word Count}: $86$ words.


\pagebreak
\tableofcontents
\pagebreak

\section{How Technology is Transforming Customer Retail: CT5: ComputersProblemSolving, Slide 21}\label{section:CT5}

During Professor Padraig Cunningham's lecture on how computers change the way we solve problems \cite{CT5}, we explored how computers have introduced new problem-solving techniques and have had a transformative role in how we approach problems. Hakken and Appel famously leveraged computational power to prove by exhaustion that four colours suffice to colour a planar graph. This resolved the long-standing four colour conjecture by proving it affirmatively, and resulted in a powerful demonstration of how computational power could be leveraged to aid in mathematical proofs \cite{appel1977} \cite{appel1977_2}. In this section, the author examines how technology is having a similarly transformative effect in the retail industry, with a particular focus on the leveraging of customer analytics.\\
\\
Evidence of retail's existence dates back to the Lydians in the seventh century BCE \cite{HoM}. Until the introduction of computers and online retail channels, consumer purchases necessarily were concentrated in physical retail stores where retailers could become familiar with individual customers and their behaviour \cite{REINARTZ2019350}. The introduction of computers in the 20th century has paved the way for a radical transformation of the retail industry \cite{REINARTZ2019350}. The 1956 patenting of the Rolodex by Neilsen \cite{Rolodex} provides an example of an approach to customer recording before the widespread usage of computers. In 1958, a paper by IBM Researcher H.P. Luhn detailed an automated Business Intelligence tool which could process documents and distribute key points of interest to relevant business executives \cite{BI}. Although due to technical limitations at the time Luhn's design is simple in comparison to solutions offered today, it demonstrates an early insight into the potential for automated business analytics tools. The introduction of 'database marketing' in the 1980s highlights how significant developments began to be made in the area of retail analytics and retail marketing, with then president of Precision Marketing Corporation Rick Courtheoux detailing in the CatalogAge 1988 roundtable how database systems allowed for the introduction of customer lifetime value (CLV) segmentation by customer type and was used to identify key customers for clientelling purposes \cite{:1988aa}. \\
\\
The popularisation of customer retention management (CRM) systems in retail in the 1990s and early 2000s served as a key step in the digital transformation of retail operations \cite{:2000aa} \cite{Feinberg:2002aa}. These systems are used by retailers to store customer data including demographic information, contactability data, and transactional data \cite{PELTIER20131}. The 1990s saw the popularisation of loyalty programs which served as a key method for retailers to gather customer data \cite{Rowley:2000aa}. Sign-up forms were used to create customer records within CRM systems, and by linking transactions involving a specific loyalty card to the customer who registered that card, retailers could begin to link customer data with transaction data in order to build a customer profile \cite{Rowley:2000aa}. This data served as the basis for 'customer analytics'; here this will be defined as "the advanced technology embedded process that assists in extracting insights about consumer behaviour from both online/mobile and offline channels to predict offerings and enhance the adaptive capacity to formulate marketing strategies" (per \cite{HOSSAIN2020102187}).\\
\\
Through the utilisation of CRM systems and the deployment of customer analytics, German et al concludes that of the industries they examined, retailers are most likely to benefit, in terms of firm performance, from increased deployment of customer analytics \cite{GERMANN2014587}. By analysing customer data and developing key customer segments for targetted coupon marketing campaigns informed by prior transactional behaviour, retailers can boost campaign revenue by $50\%$ more than would be achieved by a broader campaign targetting a wider customer base \cite{Rossi:1996aa}. The generation of customer segments by using clustering techniques to segment a retailer's customer base and using decision trees to determine the customer's final profile allows retailers to subsequently implement market basket analyses on these customer groups to present customers in these segments with affinity products on which they are more likely to convert \cite{segment-cluster}. By developing churn prediction models, retailers are able to automatically detect customers who are at risk of lapsing from their business and utilise their CRM data to deliver highly targetted marketing campaigns to customers in an attempt to maintain customer activity \cite{MIGUEIS201211250}. The adoption of CRM systems has allowed for the the development of more advanced and accurate CLV models due to a continuously expanding volume of data collection and increasing depth of historical data. This allows retailers to identify customers who are likely to provide high value to the business over the course of their lifecycle with the retailer and to closely monitor and report on these customers' performance intervening when the customer is expected to underperform on their predicted lifetime value \cite{GLADY20092062}. \\
\\
The rise of online retail and widespread adoption of social media has provided retailers who posess an online distribution channel with an even richer customer profile. Online account sign-up forms serve as an immediate entry-source into CRM systems while the range of data sources grew to include aspects such as clickstream data using tools such as Google Analytics and Google Tag Manager \cite{BRADLOW201779}, customer social media engagement data \cite{Batrinca:2015aa}, engagement with digital acquisition streams and optimising acquisition efforts in these streams (organic via search engine optimisation, email, social, video, mobile app) \cite{Walters:2015aa}. These additional data sources allow for CRM data to be richly overlaid with customer engagment behaviour to create highly personalised customer journeys, from highly tagetted campaigns to dynamic websites \cite{BRADLOW201779}.\\
\\
The advances in data availability and analytical techniques not only can aid in the supporting of business decisions and marketing strategies, but can be a powerful source of product development within retailers. One area of particular interest to retail is the development of automated personalised product recommendation tools. Zhang et al concluded that higher quality personalised product recommendation "are associated with lower consumers' product screening cost and higher consumer decision-making quality, which in turn, is positvely associated with consumer repurchase intention" \cite{10.2307/41409964}. For fashion retailers one method of developing personalised recommendation models $($as briefly touched on$)$ is to use CLV clustering combined with market basket analyses to generate product recommendations which have a positive impact on average transaction value \cite{RODRIGUES2016136}. Continuing with the context of retail fashion, an on-going area of research is in the use of deep-learning models to generate outfit predictions which can be subsequently embedded into instore or online customer-facing systems to provide automated outfit recommendations to customers and alleviating burden from merchandisers, buyers, or stylists.  Due to the complex and subjective nature of outfits, a strong set of customer and product data, is necessary to develop appealing outfits, as models can involve a combination of image analysis, text analysis of product tags, and customer data. \cite{Guan:2019aa}. \\
\\
The development of highly targetted customer campaigns and automated customer journeys to optimise engagement, revenue, and conversion; the ability to deploy complex models anticipating the risk of churn and a customer's lifetime value; and the introduction of advanced product features such as automated product recommendations and style-boards is something which is only possible in a data-rich environment. The introduction of computers and technological advancements has not only transformed how we as consumers shop via the introduction of new digital channels, but for retailers has resulted in a paradigm shift in how they approach problems and design solutions to challenges facing their business.\\
\\
\textbf{Section Word Count}: $1223$ words.\\
\textbf{Cumulative Word Count}: $1309$ words.

\pagebreak
\section{Text Analytics in Retail: CT7-What Lies Beneath, Slide 39}\label{section:CT7}
In Professor Julie Berndsen's lecture on natural language processing (NLP) \cite{CT7} we were introduced to how computers may analyse natural language, and saw how the NLP industry is anticipated to grow and be a disruptive technology within industries; in particular, a key example highlighted during this lecture was the use of NLP to understand customer sentiment during the customer journey in a retail context. In this section, the author builds on the examination of how technology has had a transformative impact on retail as described in \ref{section:CT5} to explore how text analytics and NLP is being used in retail.\\
\\
Since the development of CRM systems and the shift towards a customer-centric approach to retail operations, two key performance indicators $($KPI$)$ widely used within a retail environment are the customer satisfaction score $($CSAT$)$, primarily designed to measure satisfaction with products and services, and net promotor score $($NPS$)$ which is used as a measure of the customer journey and the impact on their brand loyalty. Both KPIs share a commonality in that they are designed to measure the customer's perception of a retailer following some form of interaction with the retailer. Typically, these perceptions are gathered through the customer's completion of a survey; this process is detailed more comprehensively by Hayes in \cite{Hayes:2008aa}, but NPS scoring is typically accomplished by asking customers to rate the statement "Would you recommend this company to a friend?" and categorising the responses into the buckets of "Detractor", "Passive", and "Promotor", with the NPS score deriving from the formula:\\
\\
\begin{math}
\text{NPS}:=\frac{\# \text{of Promotors} - \# \text{of Detractors}} {\# \text{of Respondents}}
\end{math}
\cite{1158740720031201} where $\text{NPS} \in [-1,1]$.\\
\\
While the distillation of the customer's self-reported loyalty into a single number can be valuable as a quick insight, its value as a measure of growth as initially designed has been repudiated within academic literature \cite{Keiningham:2007aa}, but remains persistent within retail and may still provide insight into the customer's experience $($although examined in a travel context, \cite{14114988820200101} provides an analysis of the value of NPS with customer sentiment$)$. Following up NPS or CSAT questions with open-ended questions is a key element of customer surveying to gain an insight into what factors are contributing to the KPI scores \cite{Hayes:2008aa}.\\
\\
As CSAT and NPS surveys may generate substantial volume of responses depending on the size of the retailer, and as manually crawling through unstructured survey responses may be a time-consuming process depending on the response volume and size of the survey, sentiment analysis can play a transformative role in helping businesses gain insight into what is driving these KPI scores in order to develop strategies to enhance service offerings and further enrich CRM data with sentiment metrics. By stripping stop words from customer responses and developing word clouds or frequency charts segmented by promotor classification $($or introducing CRM metrics to provide further segmentation$)$ this insight can be enhanced in a comparatively quick $($compared to manually examining the data$)$ manner to pinpoint key customer segments driving results \cite{10.4018/IJDWM.2019100102}.\\
\\
While NPS and CSAT may be valuable in analysing known customer behaviour, the widespread adoption of social media has served as another key channel for customer interactions, with 65 billion interactions across Instagram, Facebook, and Twitter with US brands' social media content between January and October 2015 \cite{Yuki458}. While customer interactions on social media content may provide valuable insight for brands, there are significant challenges present in filtering this volume down to relevant, actionable, human-digestible content. The linguistic differences which may be present in online communication, and completing this insight-gathering in a timely manner while the results are actionable in spite of the volume pose barriers to retailers without sufficient technological processes. One proposed approach to tackling this challenge is in the implementation of neural networks for the purpose of analysing brand sentiment which allows brands to gain insight into social media commentary in order to quickly respond to customer sentiment changes of their own brand, and to use these techniques to analyse customer perception of competitors \cite{doi:10.1080/07421222.2016.1267526} \cite{IBRAHIM201932}. \\
\\
For a retailer's own communication with customers, sentiment analysis can help creative teams within retailers to gain a deeper understanding of the copy and call-to-actions (CTAs) which are most effective in driving engagement, a prominent example of such being the Salesforce Einstein Copy Insights tool \cite{copy-insights}. By analysing the sentiments of subject lines, and identifying what terminology results in optimal open rates (OR), click-through-rates (CTR), and conversion, retailers can optimise their messaging to drive positive email KPIs \cite{Hanna:2015aa}. Traditional methods of optimising email content, timing, and segmentation include A/B Testing elements of an email such as subject line testing, in-email copy and graphic testing, and segmentation testing. Using A/B tests where different versions of one email are tested on different samples of cutomers, email creation can be slowly optimised to gradually accumulate learnings. These traditional principles of email optimisation, as outlined in \cite{HannaIn:2015aa}, still remain relevant with the recent development of using text analytics to develop insights into copy optimisation, but by combining traditional optimisation methods with advanced text analytics, retailers can quickly gather learnings and apply these with CRM insights to generate highly personalised email content which has been optimised for specific audiences, and these copy insights can be utilised throughout in guiding business creative material and tone of voice \cite{Artun:2015aa}. \\
\\
In presenting personalised content, such as personalised product recommendations, as outlined in \ref{section:CT5} text analytics is a key aspect utilised in these models to analyse product descriptions and product tags to suggest relevant products \cite{Guan:2019aa}. Product catalogs in retailers not only experience continuous growth, but contain numerous product attributes (such as description, size, division, department, title). As the product catalog grows, a key challenge is present in maintaining consistency in how products are tagged. Consistency in these areas plays a key role in developing recommendation systems. One method implemented by fashion retailer ASOS is the use of convoluted neural networks to analyse product text and generate product attribute tags in a consistent manner\cite{cardoso2018product}.\\
\\
Through usage in sentiment analysis of social media comments, the analysis of NPS and CSAT survey responses, the creation of optimised creative copy, and the role it plays not only in developing recommendation algorithms but in developing high data quality within product catalogs, text analytics and natural language processing is an area with an ever-growing number of applications within a retail environment. Through the usage of text analytics, not only can companies gain an insight into their customers' engagement data, their customers' transaction data, and their customers' demographic data, but they can also gain an understanding of customer perception and sentiment. When all of these aspects are combined, retailers can create incredibly rich customer profiles which can inform the business direction in all areas. From the usage of paper-based CRM methods such as the rolodex as discussed in \ref{CT5}, to the rich and varied profiles which retailers can build at an individual customer level, not only has technology truly transformed the retail environment but text analytics is playing a key role pushing this transformation to new heights and optimising the customer experience by helping to develop continuously improving, personalised customer journeys.\\
\\
\textbf{Section Word Count}: $1216$ words.\\
\textbf{Cumulative Word Count}: $2525$ words.

\pagebreak
\section{Ethics of Machine Learning in Retail: CT9-Machine Ethics, Slide 11}\label{section:CT9}
In \ref{section:CT5} and \ref{section:CT7} the author has explored how computers have rapidly changed the operation of retailers and how text analytics and sentiment analysis combined with these enhanced data sources provide retailers with rich customer profiles to guide their decision making. In Dr. Vivek Nallur's lecture on Machine Ethics, we saw a variety of scenarios where the deployment of artificial intelligence $($AI$)$ resulted in ethically questionable or morally unjust outcomes such as disadvantaging women or a variety of ethnicities within the hiring process, despite no bias being purposely embedded into the model's algorithm\cite{CT9}. In this section, the author examines whether it is 'right' for retailers to possess and act upon such rich customer profiles, and discusses some ethical concerns and the implications for consumers.\\
\\
Compulsive buyers are individuals who purchase products not necessarily for the acquisition of the product, but for the gratification of completing a purchase; compulsive buyers are categorised by an overwhelming desire to purchase regardless of the negative consequences which may arise including the amassing of significant debt \cite{10.1086/209204}. In the US, this disorder has a prevalence of 5.8\% in the general population \cite{Black:2007aa} and shares commonalities with other addictive disorders \cite{krych1989abnormal}. Kukar-Kinney et al found a positive correlation between compulsive buyers and a preference towards online retail channels \cite{Kukar-Kinney:2009aa}. Due to the potentially significant economic ramifications of compulsive buying, this is a highly vulnerable segment of customers. With the advanced CRM and analytics capabilities retailers possess, and the impact this can have on delivering highly personalised and targeted campaigns where offers are provided to specific customer segments to encourage conversion \cite{Rossi:1996aa} and the capabilities for dynamic pricing \cite{https://doi.org/10.1002/nav.20204} and time-sensitive discounts to which compulsive shoppers are particularly vulnerable \cite{KUKARKINNEY2016691}, in the view of the author there is a moral responsibility on the part of retailers to implement safeguards to ensure that compulsive buyers are adequately protected from questionably predatory promotional campaigns. The author has been unable to find published examples of retailers taking active measures to identify and dissuade compulsive shoppers, yet if the usage of customer analytics to target lapsed shoppers results in a recovering compulsive shopper relapsing, although the retailer may benefit financially the personal consequence to the customer could be economically and personally devastating.\\
\\
The capturing of customer data itself and the right for individual privacy is a topic which has been examined through a variety of philosophical frameworks \cite{10.2307/25123662}. The volume of customer data being held by organisations is a growing customer concern \cite{Gimpel:2018aa}. The introduction of GDPR in a European context helps to provide consumers with data protection by ensuring that customer data can only be held and processed with consumer consent or an alternative lawful basis $($as outlined within the legislation$)$ and that this consent may be revoked at the consumer's request \cite{ITGP-Privacy-Team:2017aa}. The author questions however whether consumers are fully aware of the depth of data which may be processed, and how precisely this may be utilised to develop subtle yet powerful methods to encourage them to spend. Given that there is a relatively poor understanding as to what AI is, with only 42\% of respondents to a UK survey able to even define AI \cite{10.1145/3306618.3314232}, it is unclear to the author if these concerns originate from how retailers may generate targeted campaigns, if the key driver of concern is the risk of potential data breaches, or if it is merely the nature of acquiring data which imposes on the idea of individual right to privacy \cite{:2015aa}. Indeed, Ooijen and Vrabec highlight that although customers must provide consent for data capure and although retailers must detail what their data is used for, this information is often not properly understood by consumers and, when presented with the option to consent, will simply consent when request rather than considering the consequences of providing permission for data capture \cite{Ooijen:2019aa} . If customers cannot properly understand how their data is truly used and simply consent without considering what that means, can customers truly be in control of their data as GDPR intends \cite{ITGP-Privacy-Team:2017aa}? Regardless of the exact origin of the concerns, the vast collection of data which retailers capture and the mining of this data using machine learning technique may allow retailers to accurately predict customer intent or future behaviour before the customer is even consciously aware of their intent; is it right for a retailer $($or indeed any business$)$ to, in some cases, be able to understand what actions a customer will make and guide them towards that before the customer has even consciously decided to complete that action?\\ 
\\
Concerns over data breaches are well-founded. In the US "identity theft resulted in  [$\dots$] loses of $56$ billion dollars in $2005$ with up to $35$\% of known identity thefts caused by corporate data breaches" \cite{Romanosky:2011aa}. As retailers gather increasing volumes of customer data, the risk posed by a data breach increases as more customer information is at risk of being exposed by such leaks. Not only is identity theft a potential concern but depending on the nature of the retailer, data breaches may result in sensitive information being leaked which could cause embarrassment $($e.g. browsing and purchasing sensitive beauty products or buying pharmaceutical items through online retailers$)$, and information such as political leanings, emotional responses to certain copy, or estimated salary $($which may be derived via sentiment analysis and spend analysis$)$ could reveal personal insights which should not be exposed without the customer's permission.\\
\\
As we have seen in \ref{section:CT5} and \ref{section:CT7}, the usage of customer analytics is not something which solely benefits businesses. The use of personalised product recommendations and content can result in the development of material which directly aids the experience for consumers browsing online \cite{10.2307/41409964} and deploying features which can be utilised in-store and online such as the example of automated outfits \cite{DBLP:journals/corr/abs-1904-00741} to enhance the customer retail experience. \\
\\
While retailers engage in an analytics arms race to gain advantages over competitors in customer acquisition and retention by enhancing their offerings and marketing using customer analytics \cite{https://doi.org/10.1002/pa.1944}, it is important not only to consider how data can be utilised to maximise revenue from a business perspective, but also to consider how customer analytics can be implemented in an ethical manner. The risk of inadvertently preying on vulnerable segments of customers and preventing the gathering of intrusive or extraneous data which isn't directly needed for core business outcomes should not be seen as an afterthought but must be a cornerstone in the deployment of any customer analytics strategy; in the views of this author both businesses and data scientists have a personal and moral responsibility to design systems where customer privacy and protecting vulnerable groups is a cornerstone of the design and not merely a retrospective afterthought. \\
\\
\textbf{Section Word Count}: $1180$ words.\\
\textbf{Cumulative Word Count}: $3705$ words.

\pagebreak
\bibliography{CTCOMP47340}
\bibliographystyle{plain}

\end{document}